
\documentclass[12pt]{article}
\usepackage{graphicx}
\usepackage[a4paper, width=150mm, top=25mm, bottom=25mm]{geometry}
\usepackage{fancyhdr}
\pagestyle{fancy}
\fancyhead[L]{Powerpuff Unix Shell}
\fancyhead[R]{}

\usepackage{listings}
\graphicspath{ {./images/} }
\begin{document}

 \begin{titlepage}

    \begin{center}
        \line(1, 0){300}\\
        [0.20in]
        \huge{\bfseries Powerpuff Unix Shell} \\
        \textsc{\Large The Unix Shell}\\
        \line(1, 0){250}\\
        \vspace{1mm}
        
    \end{center}
    
    \begin{center}
        \textbf{Course Title:} Operating System\\
        \textbf{Course Code:} CSE325\\
        \vspace{10mm}
        \huge{Project Report}
    \end{center}
    
    \vspace{20mm}
    \textbf{Submitted To} 
    
    \begin{center}
        Md. Ashraful Islam (Emon) \\
        Lecturer, \\
        Department of Computer Science \& Engineering, \\
        East West University, Bangladesh.
    \end{center}
    \vspace{20mm}
    \hspace{5mm}
    \textbf{Author}
    \begin{flushleft}
        \hspace{40mm}
        Shaykh Siddique(2016-1-60-053)\\ \hspace{40mm}
        Thajiba Tabassum(2016-1-60-053),\\ \hspace{40mm}
        S. M. Saikat Hossain(2016-1-60-053),\\ \hspace{40mm}
        Md.Jadid Mostafiz(2016-1-60-053),\\ \hspace{40mm}
        Shaik Md. Ibnay-Momen(2016-1-60-053),\\ \hspace{40mm}
        Jannatul Ferdous Sorna(2016-1-60-053)
    \end{flushleft}

    \end{titlepage}
    
    \section{Introduction}
    A Shell provides us with a command line interface to the Unix system. It gathers input from you and executes programs based on that input. When a program finishes executing, it displays that program's output.\\
    Shell is an environment in which we can run our commands, programs, and shell scripts. There are different flavors of a shell, just as there are different flavors of operating systems. Each flavor of shell has its own set of recognized commands and functions.\\
    Shells are mostly used to make control of remote connection of a server.

    \section{Software Description and Facilities}
    This Software is just a sample Unix Shell developed in \textbf{C}. 
    \subsection{Parsing}
    As We are working with command line, we must need to parse all the commands. A single command will be separated by semicolon ';' and in a single command - Command words and parameters are separated by space. \\ 
    We used a parsing simulation software named \textbf{ANTLR} to simulate a basic parse tree. 
    \vspace{10mm}\\   %paragraph
    \textbf{Input Commands:}
    \begin{lstlisting}[language=bash]
        cd Desktop/shell;
        gedit;
        touch file1 file2 file3;
        ls -l;
        cat doc1.txt doc3.txt;
    \end{lstlisting}
    \vspace{10mm}   %paragraph
    \textbf{Parse Tree} \\
    \includegraphics[width=\textwidth]{antlr4_parse_tree.png}
    
    \subsection{Implementation of Commands}
    \vspace{5mm}
    \textbf{cd [path]}\\
    \verb|cd| is a command of changing directory. It expects one parameter, the path of the directory to change in. Using a function called \verb|chdir(directory)|, where parameter directory is the path.
    \vspace{5mm}\\
    \textbf{ls}\\
    \verb|ls| is a command which will list all the files and directories of current working directory.
    Opening current directory, read and print all files and folder names. Using \verb|opendir(param)|, where  param is the path of directory.\\ \\
    For reading files and folder names,  \verb|readdir(dr)|, where dr is the current directory object and this function is return the names of all files and folders.
     \vspace{5mm}\\
    \textbf{mkdir [folder]}\\
    Using a function \verb|mkdir(param1, param2)|, where parameter one is the name of new folder and parameter two is the permission of this folder. Here we use the permission $(777)$ read, write, and execute.
     \vspace{5mm}\\
    \textbf{touch [file]}\\
    Creating a file just in write mode. Expecting one or more parameter.
    \vspace{5mm}\\
    \textbf{cat [file]}\\
    Open all of those files, read and print in console. Expecting one or more parameter.
    \vspace{5mm}\\
    \textbf{echo [string]}\\
    Just printing all strings. Expecting one or more parameter.
    \vspace{5mm}\\
    \textbf{cp [source] [destination]}\\
    Opening the source file, load all the data into program. Then go to destination file, create that destination file and write all data.
    \vspace{5mm}\\
    \textbf{exit}\\
    Command exit is used to terminate the program.

\end{document}}